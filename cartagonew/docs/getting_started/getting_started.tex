%
% CArtAgO guide
%
\documentclass[11pt]{report}
 \newif\ifpdf
 \ifx\pdfoutput\undefined
 \pdffalse % we are not running PDFLaTeX
 \else
 \pdfoutput=1 % we are running PDFLaTeX
 \pdftrue
 \fi
%%%%%%%%%%%%%%%
 \ifpdf
 \usepackage[pdftex]{graphicx}
 \else
 \usepackage{graphicx}
 \fi
%%%%%%%%%%%%%%%
 \ifpdf
 \DeclareGraphicsExtensions{.pdf, .jpg, .tif}
 \else
 \DeclareGraphicsExtensions{.eps, .jpg}
 \fi
%%%%%%%%%%%%%%%

\newcommand\xc[1]{\chaptername~\ref{chap:#1}}
\newcommand\labelchap[1]{\label{chap:#1}}
\newcommand\xa[1]{\appendixname~\ref{app:#1}}
\newcommand\labelsec[1]{\label{sec:#1}}
\newcommand\xs[1]{\sectionname~\ref{sec:#1}}
\newcommand\xsp[1]{\sectionname~\ref{sec:#1} \onpagename~\pageref{sec:#1}}
\newcommand\labelssec[1]{\label{ssec:#1}}
\newcommand\xss[1]{\subsectionname~\ref{ssec:#1}}
\newcommand\xssp[1]{\subsectionname~\ref{ssec:#1} \onpagename~\pageref{ssec:#1}}
\newcommand\labelsssec[1]{\label{sssec:#1}}
\newcommand\xsss[1]{\subsectionname~\ref{sssec:#1}}
\newcommand\xsssp[1]{\subsectionname~\ref{sssec:#1} \onpagename~\pageref{sssec:#1}}
\newcommand\labelfig[1]{\label{fig:#1}}
\newcommand\xf[1]{\figurename~\ref{fig:#1}}
\newcommand\xff[2]{\figurenames~\ref{fig:#1}~and~\ref{fig:#2}}
\newcommand\xfp[1]{\figurename~\ref{fig:#1} \onpagename~\pageref{fig:#1}}
\newcommand\labeltab[1]{\label{tb:#1}}
\newcommand\xt[1]{\tablename~\ref{tb:#1}}
\newcommand\xtt[2]{\tablenames~\ref{tb:#1}~and~\ref{ab:#2}}
\newcommand\xtp[1]{\tablename~\ref{tb:#1} \onpagename~\pageref{tb:#1}}
\newcommand\labelenum[1]{\label{enum:#1}}
\newcommand\xen[1]{(\ref{enum:#1})}
\newcommand\xenp[1]{(\ref{enum:#1}) \onpagename~\pageref{enum:#1}}
%\newcommand{\chaptername}{Chapter}
%\newcommand{\figurename}{Figure}
%\newcommand{\tablename}{Table}
\newcommand{\sectionname}{Section}

%******************************************************************************%
\newcommand\note[1]{NOTE:\emph{#1}}
\newcommand\tbc[1]{TO BE COMPLETED: \emph{#1}}
\newcommand\outd[1]{OUTDATED: \emph{#1}}
\newcommand\tbd[1]{TO BE COMPLETED: \emph{#1}}
\newcommand\code[1]{{\small{\mbox{\texttt{{#1}}}}}}
\newcommand\sym[1]{{\small{\mbox{\textit{{#1}}}}}}
\newcommand\ttit[1]{\texttt{\textit{#1}}}

\newcommand\version[1]{\mbox{Document revision: #1}}
\newcommand\approvedby[1]{\mbox{Approved by: #1}}
\newcommand\receivedby[1]{\mbox{Received by: #1}}
\newcommand\creationdate[1]{\mbox{Creation date: #1}}
\newcommand\repauthor[1]{\mbox{Main author: #1}}
\newcommand\lastchangesdate[1]{\mbox{Last Changes date: #1}}
\newcommand\noa[2]{\noindent\emph{Note of the author (#1): }#2\\\\}
\newcommand\logo{
    \begin{figure}[tp]
        \begin{center}
            \inputgraphics[width=4cm]{../shared/logo}
        \end{center}
\end{figure}}

%******************************************************************************%
\newcommand{\cartagoversion}{2.0}
\newcommand{\jaca}{\mbox{\sf{JaCa}}}
\newcommand{\jason}{\mbox{\sf{\emph{{Jason}}}}}
\newcommand{\AandA}{\mbox{\sf{{A\&A~}}}}
\newcommand{\cartago}{\mbox{\sf{CArtAgO}}}
%******************************************************************************%


\title{{\huge{\bf{{\cartago} Getting Started}}\\~\\\mbox{version: \cartagoversion}\\\mbox{~}\\}
{\small{
   \repauthor{aricci}  \\  
    \creationdate{20100801}\\
    \lastchangesdate{20100801}\\
    % \receivedby{aricci}\\
    % \approvedby{aricci}\\
    }}
}

\author{DEIS, Universit\`{a} di Bologna, Italy}

\date{}

\begin{document}

\maketitle
\sloppy

\chapter{Getting Started}

\section{{\cartago} Distribution}

{\cartago} 2.0 distribution contains the source code and the binaries of the {\cartago} framework infrastructure, including also bridges to agent programming platforms available by default -- currently, for version 2.0, only the {\jason} bridge is available.


The distribution includes the following folders:
%
\begin{itemize}
\item \code{src}
\begin{itemize}
\item \code{main} -- this folder contains the source code of {\cartago}
\item \code{bridges} -- this folder contains the source code of available bridges to use {\cartago} with Agent Programming Languages
\end{itemize}
\item \code{lib} -- this folder contains the Java libraries (JAR) containing the binaries for {\cartago} and available bridges
\item \code{examples}
\begin{itemize}
\item \code{main} -- this folder contains examples related to {\cartago}, with agents developed using the simple Java API
\item \code{bridges} -- this folder contains examples using agents written using existing agent programming languages 
\end{itemize}
\item \code{doc}
\begin{itemize}
\item \code{getting\_started.pdf} -- this document
\item \code{tech-rep-jaamas-2010.pdf} -- a technical report describing the environment programming idea and {\cartago} model
\item \code{cartago\_by\_example.pdf} -- this is a tutorial describing the main features of {\cartago} using {\jason} as APL
\item \code{main-api} -- this folder contains the Java doc of {\cartago} main sources
\item \code{c4jason-api} -- this folder contains the Java doc of the source code of the bridge {\jason}+{\cartago}
\end{itemize}
\end{itemize}

\section{Using {\cartago}}

\subsection{Using {\cartago} from {\jason}}

To use {\cartago} into {\jason} applications, just include in the MAS configuration file:
\begin{itemize} 
\item \code{cartago.jar} and \code{c4jason.jar} libraries in the declared class path; 
\item the declaration of {\code{c4jason.Environment}} as environment;
\item the declaration of \code{c4jason.CAgentArch} as agent architecture, for each agent that needs to work within {\cartago} environment.
\end{itemize}
%
\noindent For example, consider the first example included in \code{examples/bridges/jason}, \code{example00-hello-world.mas2j}:
 %
{\small{
\begin{verbatim}
MAS example00_hello_world {
   
  environment: 
  c4jason.CartagoEnvironment
    
  agents:  
  hello_agent agentArchClass c4jason.CAgentArch;

  classpath: "../../../lib/cartago.jar";"../../../lib/c4jason.jar";    
}
\end{verbatim}}}

%
\noindent By default, a {\cartago} node is installed and the specified agents are automatically joined to the default workspace.

\subsection{Using {\cartago} in {Java}}

To use {\cartago} with simple Java agents,  just include {\code{cartago.jar}} in the classpath.
%
Then, \code{CartagoService} class provides the basic API to start a node and open a working session inside.
%
By starting a session (method \code{startSession}), an object implementing the \code{ICartagoSession} interface is returned,
to act inside the workspace.
%
For instance, consider the first example included in \code{examples/main/examples}, \code{Ex00a\_HelloWorld}:
%
{\small{
\begin{verbatim}
package examples;

import cartago.*;
import cartago.security.*;

public class Ex00a_HelloWorld {
	
  public static void main(String[] args) throws Exception {		
    CartagoService.startNode();
    ICartagoSession session = CartagoService.startSession("default", 
                   new AgentIdCredential("agent-0"), null);
    session.doAction(new Op("println","Hello, world!"), null, -1);
  }
}
\end{verbatim}}}


\section{{\cartago} sources on SVN}

The up-to-date version of {\cartago} sources is available on SVN repository on Source Forge.
% 
The address of the repository is: \code{http://cartago.svn.sourceforge.net/svnroot/cartago}

\end{document}
